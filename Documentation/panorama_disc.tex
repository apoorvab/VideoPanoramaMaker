\subsection{Panorama Discovery}

Panorama discovery is defined as finding frames in a video that constitutes a panorama.  
Videos may contain frames that are not panoramic source images.  
We consider a video segment with the following properties to be a valid panorama source.  
A video segment is defined as a series of sequential frames.  There may be multiple video segments within a shot.

\vspace{5 mm}

\noindent\textbf{Property 1}: A video segment should cover a wide field-of-view based on the definition of panorama imagery.

\vspace{5 mm}

\noindent\textbf{Property 2}: A video segment should be �mosaicable�.  
Specifically, the underlying camera motion between frames should observe a certain camera motion model.  
We use a homography to model the motion between frames.  This property should be met after the shot detection algorithm.

\vspace{5 mm}

\noindent\textbf{Property 3}: The frames in a video segment should have high image quality.  
This conservative strategy is adopted so we can ignore methods to improve image visual quality, such as de-blurring and de-blocking.

\vspace{5 mm}

Ideally, we want to discover panoramas that have a very wide field-of-view and are of very high visual quality.  
Unfortunately, there is a tradeoff in practice.  More frames need to be included in order to have a wider field-of-view.  
However, including more source frames can often degrade the visual quality due to motion estimation error.  
We formulate discovering panoramas from a video shot as an optimization problem.  
The optimization problem aims to find a series of video segments that achieve an optimal balance between maximizing 
the visual quality of the resulting panoramas and maximizing the scenes they cover.  We implemented a variation
of the original optimization problem presented in Feng [1].  An overview of the orginal optimization problem is
described below followed by a detailed explaination of our implemented approximation.

\subsubsection{Original Optimization Problem}

The original optimization problem presented in Feng [1] is shown in (\ref{eq:OrgOptEqu}).
Directly solving the constrained nonlinear programming problem is difficult.  
In fact the authors of the paper approximated the optimal solution.  Details of 
their approximation is discussed in the paper [1].


% OPTIMIZATION PROBLEM
% FIRST LINE
\begin{equation} 
\hat{S}=arg min_{S}
\{ \sum_{S_{i}\epsilon S} E_{v}(S_{i}) + \sum_{S_{i},S_{j}\epsilon S} E_{a}(S_{i},S_{j}) \} \label{eq:OrgOptEqu}
\end{equation}
% SECOND LINE
\begin{equation*}
where \; S=\{ S_{i} \mid S_{i} \subseteq V \}, \;
\forall S_{i},S_{j} \epsilon S, \;
S_{i} \cap S_{j} = {\o}
\end{equation*}
% THIRD LINE
\begin{equation*}
s.t. \left\{ \begin{array}{l}
E_{v}(S_{i}) < \delta , \forall S_{i} \epsilon V \\
\varepsilon(S_{i}) > \beta A , \forall S_{i} \epsilon V
\end{array} \right.
\nonumber
\end{equation*}

\vspace{5 mm}

$S$ denotes a set that contains non-overlapping segments $S_{i}$ of the video clip $V$.
$E_{v}(S_{i})$ is the visual quality cost of stitching a panorama from $S_{i}$, and
$E_{a}(S_{i},S_{j})$ is the cost of splitting a panorama from $S_{i} \cup S_{j}$ into smaller
ones from $S_{i}$ and $S_{j}$ respectively.  $\varepsilon(S_{i})$ denotes the extent of the scene
in $S_{i}$, which is required to be bigger than $\beta$ times the original video frame
size $A$.  To gaurantee the visual quality of the panorama, the visual quality distortion
$E_{v}(S_{i})$ is required to be less than a threshold $\delta$.

\subsubsection{Implemented Approximation}

We relaxed the optimization problem as shown in equation (\ref{eq:OrgOptEqu}) to the two step problem shown in equations (\ref{eq:OurOptEqu1}) and (\ref{eq:OurOptEqu2}).
A major difference is that we consider overlapping segments $S_{i}$ of the video clip $V$ as part of the set $S$.  This difference
provides our final algorithm with two major benefits.  The first is that $E_{a}(S_{i},S_{j})$, the cost of splitting a panorama, 
is no longer considered, which reduces the computational difficulty.  The second is that the user has another tunable parameter $\alpha$ 
to achieve the desired results.  This is a benefit rather than a disadvantage because to some extent identifying panoramas is
subjective to the user.  This will be discussed in greater detail in the Experiments section.

\vspace{5 mm}

\noindent\textbf{Step 1:}
% FIRST LINE
\begin{equation}
\hat{S}=arg min_{S} \{ \sum_{S_{i}\epsilon S} E_{v}(S_{i}) \} \label{eq:OurOptEqu1}
\end{equation}
% SECOND LINE
\begin{equation*}
where \; S=\{ S_{i} \mid S_{i} \subseteq V \} 
\end{equation*}
% FOURTH LINE
\begin{equation*}
s.t. \left\{ \begin{array}{l}
E_{v}(S_{i}) < \delta , \; \forall \; S_{i} \epsilon V \\
\varepsilon(S_{i}) > \beta A , \; \forall \; S_{i} \epsilon V
\end{array} \right.
\end{equation*}

\vspace{5 mm}

\noindent\textbf{Step 2:}
% FIRST LINE
\begin{equation}
P = 2^{S} \label{eq:OurOptEqu2}
\end{equation}
% SECOND LINE
\begin{equation*}
s.t. \{ \varepsilon(P_{i} \cap P_{j}) < min(\alpha \varepsilon(P_{i}),\alpha \varepsilon(P_{j})), \; \forall \; P_{i},P_{j} \epsilon V
\end{equation*}

\vspace{5 mm}

$S$ denotes a set that may contain overlapping segments $S_{i}$ of the video clip $V$.
$E_{v}(S_{i})$ is the visual quality cost of stitching a panorama from $S_{i}$.
$\varepsilon(S_{i})$ denotes the extent of the scene in $S_{i}$, which is required to be bigger than $\beta$ times the original video frame
size $A$.  To gaurantee the visual quality of the panorama, the visual quality distortion
$E_{v}(S_{i})$ is required to be less than a threshold $\delta$.
$P$ is the power set of $S$ that meets the constraint of equation (\ref{eq:OurOptEqu2}).  The constraint is that for any two subsets $P_{i},P_{j}$
of $P$, the area of their intersection must be less than $\alpha \varepsilon(P_{i})$ and $\alpha \varepsilon(P_{j})$.
P is the final set of panoramas, which may contain overlapping segments.

As mentioned, $E_{v}(S_{i})$ is the visual quality cost of stitching a panorama from $S_{i}$.  This is measured from two
aspects: the incorrectness of the motion model denoted as $E_{vm}(S_{i})$, and the source image visual quality distortion
denoted as $E_{vv}(S_{i})$.

\begin{equation}
E_{v}(S_{i}) = \alpha_{m} E_{vm}(S_{i}) + \alpha_{v} E_{vv}(S_{i}) \label{eq:Ev}
\end{equation}

\noindent where $\alpha_{m}$ and $\alpha_{v}$ are weights, with default values 1.0 and 1.0.

The motion model usually cannot be perfectly accurate when describing the correspondence between two frames.  Since we use
a homography for our motion model, homography error is used to measure the incorrectness of the model.  $E_{vm}(S_{i})$ is 
the sum of the homography errors between adjacent frames in $S_{i}$ as shown in equation (\ref{eq:Evm}).

\begin{equation}
E_{vm}(S_{i}) = \sum_{I_{k},I_{k+1} \epsilon S_{i}} H_{error}(I_{k},I_{k+1}) \label{eq:Evm}
\end{equation}

\noindent where $I_{k}$ is a frame and $H_{error}$ is the homography error between two adjacent frames.











